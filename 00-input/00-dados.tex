%%%%%%%%%%%%%%%%%%%%%%%%%%%%%%%%%%%%%%%%%%%%%%%%%%%%%%%
% Arquivo para entrada de dados para a parte pré textual
%%%%%%%%%%%%%%%%%%%%%%%%%%%%%%%%%%%%%%%%%%%%%%%%%%%%%%%
% 
% Basta digitar as informações indicidas, no formato 
% apresentado.
%
%%%%%%%
% Os dados solicitados são, na ordem:
%
% tipo do trabalho
% componentes do trabalho 
% título do trabalho
% nome do autor
% local 
% data (ano com 4 dígitos)
% orientador(a)
% coorientador(a)(as)(es)
% arquivo com dados bibliográficos
% instituição
% setor
% programa de pós gradução
% curso
% preambulo
% data defesa
% CDU
% errata
% assinaturas - termo de aprovação
% resumos & palavras chave
% agradecimentos
% dedicatoria
% epígrafe


% Informações de dados para CAPA e FOLHA DE ROSTO
%----------------------------------------------------------------------------- 
\tipotrabalho{Dissertação}

% Marcar Sim para as partes que irão compor o documento pdf
%----------------------------------------------------------------------------- 
 \providecommand{\terCapa}{Nao}
 \providecommand{\terFolhaRosto}{Nao}
 \providecommand{\terTermoAprovacao}{Nao}
 \providecommand{\terDedicatoria}{Nao}
 \providecommand{\terFichaCatalografica}{Nao}
 \providecommand{\terEpigrafe}{Nao}
 \providecommand{\terAgradecimentos}{Nao}
 \providecommand{\terErrata}{Nao}
 \providecommand{\terListaFiguras}{Sim}
 \providecommand{\terListaTabelas}{Sim}
 \providecommand{\terSiglasAbrev}{Nao}
 \providecommand{\terResumos}{Nao}
 \providecommand{\terSumario}{Sim}
 \providecommand{\terAnexo}{Nao}
 \providecommand{\terApendice}{Nao}
 \providecommand{\terIndiceR}{Nao}
%----------------------------------------------------------------------------- 

\titulo{Análises dos efeitos das variações entre Spread ex Ante e ex Post na rentabilidade dos bancos}
\autor{}
\local{Curitiba}
\data{2020} %Apenas ano 4 dígitos

% Orientador ou Orientadora
\orientador{}
%Prof Emílio Eiji Kavamura, MSc}
\orientadora{
Prof\textordfeminine~Dra. Mayla Costa, DSc}
% Pode haver apenas uma orientadora ou um orientador
% Se houver os dois prevalece o feminino.

% Em termos de coorientação, podem haver até quatro neste modelo
% Sendo 2 mulhere e 2 homens.
% Coorientador ou Coorientadora
\coorientador{}%Prof Morgan Freeman, DSc}
\coorientadora{}

% Segundo Coorientador ou Segunda Coorientadora
\scoorientador{}
%Prof Jack Nicholson, DEng}
\scoorientadora{}
%Prof\textordfeminine~Ingrid Bergman, DEng}
% ----------------------------------------------------------
%\addbibresource{10-references/referencias}

% ----------------------------------------------------------
\instituicao{Universidade Federal do Paraná}

\def \ImprimirSetor{Departamento de Economia}%
%Setor de Tecnologia}

\def \ImprimirProgramaPos{Programa Profissional de Pós-Graduação em Economia}

\def \ImprimirCurso{Mestrado Profissional em Economia}

\preambulo{
Trabalho apresentado como requisito parcial para a obtenção do título de Mestre Profisisonal em Economia no curso de Mestrado Profissional em Economia pelo Departamento de Economia da Universidade Federal do Paraná}
%do grau de Bacharel em Expressão Gráfica no curso de Expressão Gráfica, Setor de Exatas da Universidade Federal do Paraná}

%----------------------------------------------------------------------------- 

\newcommand{\imprimirCurso}{}
%Programa de P\'os Gradua\c{c}\~ao em Engenharia da Constru\c{c}\~ao Civil}

\newcommand{\imprimirDataDefesa}{
31 de Dezembro de 2020}

\newcommand{\imprimircdu}{
02:141:005.7}

% ----------------------------------------------------------
\newcommand{\imprimirerrata}{

\vspace{\onelineskip}


\begin{table}[htb]
\center
\footnotesize
\begin{tabular}{|p{1.4cm}|p{1cm}|p{3cm}|p{3cm}|}
  \hline
   \textbf{Folha} & \textbf{Linha}  & \textbf{Onde se lê}  & \textbf{Leia-se}  \\
    \hline
    1 & 10 & auto-conclavo & autoconclavo\\
   \hline
\end{tabular}
\end{table}}

% Comandos de dados - Data da apresentação
\providecommand{\imprimirdataapresentacaoRotulo}{}
\providecommand{\imprimirdataapresentacao}{}
\newcommand{\dataapresentacao}[2][\dataapresentacaoname]{\renewcommand{\dataapresentacao}{#2}}

% Comandos de dados - Nome do Curso
\providecommand{\imprimirnomedocursoRotulo}{}
\providecommand{\imprimirnomedocurso}{}
\newcommand{\nomedocurso}[2][\nomedocursoname]
  {\renewcommand{\imprimirnomedocursoRotulo}{#1}
\renewcommand{\imprimirnomedocurso}{#2}}


% ----------------------------------------------------------
\newcommand{\AssinaAprovacao}{

\assinatura{%\textbf
   {Professora} \\ UFPR}
   \assinatura{%\textbf
   {Professora} \\ ENSEADE}
   \assinatura{%\textbf
   {Professora} \\ TIT}
   %\assinatura{%\textbf{Professor} \\ Convidado 4}
      
   \begin{center}
    \vspace*{0.5cm}
    %{\large\imprimirlocal}
    %\par
    %{\large\imprimirdata}
    \imprimirlocal, \imprimirDataDefesa.
    \vspace*{1cm}
  \end{center}
  }
  
% ----------------------------------------------------------
%\newcommand{\Errata}{%\color{blue}
%Elemento opcional da \textcite[4.2.1.2]{NBR14724:2011}. Exemplo:
%}

% ----------------------------------------------------------
\newcommand{\EpigrafeTexto}{%\color{blue}
\textit{Texto}
}

% ----------------------------------------------------------
\newcommand{\ResumoTexto}{%\color{blue}

}

\newcommand{\PalavraschaveTexto}{%\color{blue}
latex. abntex. editoração de texto.}

% ----------------------------------------------------------
\newcommand{\AbstractTexto}{%\color{blue}
This is the english abstract.
}
% ---
\newcommand{\KeywordsTexto}{%\color{blue}
latex. abntex. text editoration.
}

% ----------------------------------------------------------
\newcommand{\Resume}
{%\color{blue}
Il s'agit d'un résumé en français.
} 
% ---
\newcommand{\Motscles}
{%\color{blue}
 latex. abntex. publication de textes.
}

% ----------------------------------------------------------
\newcommand{\Resumen}
{%\color{blue}
Este es el resumen en español.
}
% ---
\newcommand{\Palabrasclave}
{%\color{blue}
latex. abntex. publicación de textos.
}

% ----------------------------------------------------------
\newcommand{\AgradecimentosTexto}{%\color{blue}

}

% ----------------------------------------------------------
\newcommand{\DedicatoriaTexto}{%\color{blue}
\textit{}
	}

