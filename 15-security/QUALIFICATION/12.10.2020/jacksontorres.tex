\PassOptionsToPackage{unicode=true}{hyperref} % options for packages loaded elsewhere
\PassOptionsToPackage{hyphens}{url}
%
\documentclass[12pt,openright,oneside,a4paper,chapter=TITLE,section=TITLE,subsection=Title,english,french,spanish,portugues,sumario=tradicional]{04-class-files/abntex2}
\usepackage{lmodern}
\usepackage{amssymb,amsmath}
\usepackage{ifxetex,ifluatex}
\usepackage{fixltx2e} % provides \textsubscript
\ifnum 0\ifxetex 1\fi\ifluatex 1\fi=0 % if pdftex
  \usepackage[T1]{fontenc}
  \usepackage[utf8]{inputenc}
  \usepackage{textcomp} % provides euro and other symbols
\else % if luatex or xelatex
  \usepackage{unicode-math}
  \defaultfontfeatures{Ligatures=TeX,Scale=MatchLowercase}
\fi
% use upquote if available, for straight quotes in verbatim environments
\IfFileExists{upquote.sty}{\usepackage{upquote}}{}
% use microtype if available
\IfFileExists{microtype.sty}{%
\usepackage[]{microtype}
\UseMicrotypeSet[protrusion]{basicmath} % disable protrusion for tt fonts
}{}
\IfFileExists{parskip.sty}{%
\usepackage{parskip}
}{% else
\setlength{\parindent}{0pt}
\setlength{\parskip}{6pt plus 2pt minus 1pt}
}
\usepackage{hyperref}
\hypersetup{
            pdfauthor={Jackson da Silva Torres},
            pdfborder={0 0 0},
            breaklinks=true}
\urlstyle{same}  % don't use monospace font for urls
\usepackage{longtable,booktabs}
% Fix footnotes in tables (requires footnote package)
\IfFileExists{footnote.sty}{\usepackage{footnote}\makesavenoteenv{longtable}}{}
\usepackage{graphicx,grffile}
\makeatletter
\def\maxwidth{\ifdim\Gin@nat@width>\linewidth\linewidth\else\Gin@nat@width\fi}
\def\maxheight{\ifdim\Gin@nat@height>\textheight\textheight\else\Gin@nat@height\fi}
\makeatother
% Scale images if necessary, so that they will not overflow the page
% margins by default, and it is still possible to overwrite the defaults
% using explicit options in \includegraphics[width, height, ...]{}
\setkeys{Gin}{width=\maxwidth,height=\maxheight,keepaspectratio}
\setlength{\emergencystretch}{3em}  % prevent overfull lines
\providecommand{\tightlist}{%
  \setlength{\itemsep}{0pt}\setlength{\parskip}{0pt}}
\setcounter{secnumdepth}{5}
% Redefines (sub)paragraphs to behave more like sections
\ifx\paragraph\undefined\else
\let\oldparagraph\paragraph
\renewcommand{\paragraph}[1]{\oldparagraph{#1}\mbox{}}
\fi
\ifx\subparagraph\undefined\else
\let\oldsubparagraph\subparagraph
\renewcommand{\subparagraph}[1]{\oldsubparagraph{#1}\mbox{}}
\fi

% set default figure placement to htbp
\makeatletter
\def\fps@figure{htbp}
\makeatother

%%%%%%%%%%%%%%%%%%%%%%%%%%%%%%%%%%%%%%%%%%%%%%%%%%%%%%%
% Arquivo para entrada de dados para a parte pré textual
%%%%%%%%%%%%%%%%%%%%%%%%%%%%%%%%%%%%%%%%%%%%%%%%%%%%%%%
% 
% Basta digitar as informações indicidas, no formato 
% apresentado.
%
%%%%%%%
% Os dados solicitados são, na ordem:
%
% tipo do trabalho
% componentes do trabalho 
% título do trabalho
% nome do autor
% local 
% data (ano com 4 dígitos)
% orientador(a)
% coorientador(a)(as)(es)
% arquivo com dados bibliográficos
% instituição
% setor
% programa de pós gradução
% curso
% preambulo
% data defesa
% CDU
% errata
% assinaturas - termo de aprovação
% resumos & palavras chave
% agradecimentos
% dedicatoria
% epígrafe


% Informações de dados para CAPA e FOLHA DE ROSTO
%----------------------------------------------------------------------------- 
\tipotrabalho{Dissertação}



% Marcar Sim para as partes que irão compor o documento pdf
%----------------------------------------------------------------------------- 
 \providecommand{\terCapa}{Nao}
 \providecommand{\terFolhaRosto}{Nao}
 \providecommand{\terTermoAprovacao}{Nao}
 \providecommand{\terDedicatoria}{Nao}
 \providecommand{\terFichaCatalografica}{Nao}
 \providecommand{\terEpigrafe}{Nao}
 \providecommand{\terAgradecimentos}{Nao}
 \providecommand{\terErrata}{Nao}
 \providecommand{\terListaFiguras}{Nao}
 \providecommand{\terListaTabelas}{Nao}
 \providecommand{\terSiglasAbrev}{Nao}
 \providecommand{\terResumos}{Nao}
 \providecommand{\terSumario}{Sim}
 \providecommand{\terAnexo}{Nao}
 \providecommand{\terApendice}{Nao}
 \providecommand{\terIndiceR}{Nao}
%----------------------------------------------------------------------------- 

\titulo{Análises dos efeitos das variações entre Spread ex Ante e ex Post na rentabilidade dos bancos}
\autor{}
\local{Curitiba}
\data{2020} %Apenas ano 4 dígitos

% Orientador ou Orientadora
\orientador{}
%Prof Emílio Eiji Kavamura, MSc}
\orientadora{
Prof\textordfeminine~Dra. Mayla Costa, DSc}
% Pode haver apenas uma orientadora ou um orientador
% Se houver os dois prevalece o feminino.

% Em termos de coorientação, podem haver até quatro neste modelo
% Sendo 2 mulhere e 2 homens.
% Coorientador ou Coorientadora
\coorientador{}%Prof Morgan Freeman, DSc}
\coorientadora{}

% Segundo Coorientador ou Segunda Coorientadora
\scoorientador{}
%Prof Jack Nicholson, DEng}
\scoorientadora{}
%Prof\textordfeminine~Ingrid Bergman, DEng}
% ----------------------------------------------------------
%\addbibresource{10-references/referencias}

% ----------------------------------------------------------
\instituicao{Universidade Federal do Paraná}

\def \ImprimirSetor{Departamento de Economia}%
%Setor de Tecnologia}

\def \ImprimirProgramaPos{Programa Profissional de Pós-Graduação em Economia}

\def \ImprimirCurso{Mestrado Profissional em Economia}

\preambulo{
Trabalho apresentado como requisito parcial para a obtenção do título de Mestre Profisisonal em Economia no curso de Mestrado Profissional em Economia pelo Departamento de Economia da Universidade Federal do Paraná}
%do grau de Bacharel em Expressão Gráfica no curso de Expressão Gráfica, Setor de Exatas da Universidade Federal do Paraná}

%----------------------------------------------------------------------------- 

\newcommand{\imprimirCurso}{}
%Programa de P\'os Gradua\c{c}\~ao em Engenharia da Constru\c{c}\~ao Civil}

\newcommand{\imprimirDataDefesa}{
31 de Dezembro de 2020}

\newcommand{\imprimircdu}{
02:141:005.7}

% ----------------------------------------------------------
\newcommand{\imprimirerrata}{

\vspace{\onelineskip}


\begin{table}[htb]
\center
\footnotesize
\begin{tabular}{|p{1.4cm}|p{1cm}|p{3cm}|p{3cm}|}
  \hline
   \textbf{Folha} & \textbf{Linha}  & \textbf{Onde se lê}  & \textbf{Leia-se}  \\
    \hline
    1 & 10 & auto-conclavo & autoconclavo\\
   \hline
\end{tabular}
\end{table}}

% Comandos de dados - Data da apresentação
\providecommand{\imprimirdataapresentacaoRotulo}{}
\providecommand{\imprimirdataapresentacao}{}
\newcommand{\dataapresentacao}[2][\dataapresentacaoname]{\renewcommand{\dataapresentacao}{#2}}

% Comandos de dados - Nome do Curso
\providecommand{\imprimirnomedocursoRotulo}{}
\providecommand{\imprimirnomedocurso}{}
\newcommand{\nomedocurso}[2][\nomedocursoname]
  {\renewcommand{\imprimirnomedocursoRotulo}{#1}
\renewcommand{\imprimirnomedocurso}{#2}}


% ----------------------------------------------------------
\newcommand{\AssinaAprovacao}{

\assinatura{%\textbf
   {Professora} \\ UFPR}
   \assinatura{%\textbf
   {Professora} \\ ENSEADE}
   \assinatura{%\textbf
   {Professora} \\ TIT}
   %\assinatura{%\textbf{Professor} \\ Convidado 4}
      
   \begin{center}
    \vspace*{0.5cm}
    %{\large\imprimirlocal}
    %\par
    %{\large\imprimirdata}
    \imprimirlocal, \imprimirDataDefesa.
    \vspace*{1cm}
  \end{center}
  }
  
% ----------------------------------------------------------
%\newcommand{\Errata}{%\color{blue}
%Elemento opcional da \textcite[4.2.1.2]{NBR14724:2011}. Exemplo:
%}

% ----------------------------------------------------------
\newcommand{\EpigrafeTexto}{%\color{blue}
\textit{Texto}
}

% ----------------------------------------------------------
\newcommand{\ResumoTexto}{%\color{blue}

}

\newcommand{\PalavraschaveTexto}{%\color{blue}
latex. abntex. editoração de texto.}

% ----------------------------------------------------------
\newcommand{\AbstractTexto}{%\color{blue}
This is the english abstract.
}
% ---
\newcommand{\KeywordsTexto}{%\color{blue}
latex. abntex. text editoration.
}

% ----------------------------------------------------------
\newcommand{\Resume}
{%\color{blue}
Il s'agit d'un résumé en français.
} 
% ---
\newcommand{\Motscles}
{%\color{blue}
 latex. abntex. publication de textes.
}

% ----------------------------------------------------------
\newcommand{\Resumen}
{%\color{blue}
Este es el resumen en español.
}
% ---
\newcommand{\Palabrasclave}
{%\color{blue}
latex. abntex. publicación de textos.
}

% ----------------------------------------------------------
\newcommand{\AgradecimentosTexto}{%\color{blue}

}

% ----------------------------------------------------------
\newcommand{\DedicatoriaTexto}{%\color{blue}
\textit{}
	}


\input{09-packages/00-pacotes}
\makeindex
\usepackage{helvet}
\renewcommand{\familydefault}{\sfdefault}
\DeclareUnicodeCharacter{0301}{******}
\DeclareUnicodeCharacter{0303}{******}
\DeclareUnicodeCharacter{0327}{******}
\usepackage{booktabs}
\usepackage{longtable}
\usepackage{array}
\usepackage{multirow}
\usepackage{wrapfig}
\usepackage{float}
\usepackage{colortbl}
\usepackage{pdflscape}
\usepackage{tabu}
\usepackage{threeparttable}
\usepackage{threeparttablex}
\usepackage[normalem]{ulem}
\usepackage{makecell}
\usepackage[style=abnt,]{biblatex}
\addbibresource{10-references/referencias.bib}

\author{Jackson da Silva Torres}
\date{2020}

\begin{document}

\ifthenelse{\equal{\terCapa}{Sim}}{
\imprimircapa}{}

\ifthenelse{\equal{\terFolhaRosto}{Sim}}{
\imprimirfolhaderosto*}{}

\ifthenelse{\equal{\terFichaCatalografica}{Sim}}
 {\insereFichaCatalografica{}\cleardoublepage}
 {}

\ifthenelse{\equal{\terErrata}{Sim}}
 {\begin{errata}%\color{blue}
   \imprimirerrata
  \end{errata}}
 {}

\ifthenelse{\equal{\terTermoAprovacao}{Sim}}{
\insereAprovacao}{}

\ifthenelse{\equal{\terDedicatoria}{Sim}}{
\begin{dedicatoria}
   \vspace*{\fill}
   \centering
   \noindent
   \DedicatoriaTexto
   \vspace*{\fill}
\end{dedicatoria}
}{}

\ifthenelse{\equal{\terAgradecimentos}{Sim}}
 {\begin{agradecimentos}
    \AgradecimentosTexto
  \end{agradecimentos}
  }{}

\ifthenelse{\equal{\terEpigrafe}{Sim}}{
\begin{epigrafe}
    \vspace*{\fill}
	\begin{flushright}
        \EpigrafeTexto
	\end{flushright}
\end{epigrafe}
}{}

\ifthenelse{\equal{\terResumos}{Sim}}{
\begin{resumo}
    \ResumoTexto
    

   \noindent 
   \textbf{Palavras-chaves}: \PalavraschaveTexto
\end{resumo}

\begin{resumo}[ABSTRACT]
 \begin{otherlanguage*}{english}
   \AbstractTexto
   
   \noindent 
   \textbf{Key-words}: \KeywordsTexto
 \end{otherlanguage*}
\end{resumo}



\ifthenelse{\equal{\Resume}{}}
{}
{
 \begin{resumo}[RESUME]%Résumé
  \begin{otherlanguage*}{french}
     \Resume
     
   \noindent      
    \textbf{Mots clés}: \Motscles
  \end{otherlanguage*}
 \end{resumo}
} 


\ifthenelse{\equal{\Resume}{}}{}
{ \begin{resumo}[RESUMEN]
  \begin{otherlanguage*}{spanish}
    \Resumen 
   
   \noindent    
    \textbf{Palabras clave}: \Palabrasclave
  \end{otherlanguage*}
 \end{resumo}
}
}{}

\ifthenelse{\equal{\terListaFiguras}{Sim}}{
\pdfbookmark[0]{\listfigurename}{lof}
\listoffigures*
\cleardoublepage
}{}

\ifthenelse{\equal{\terListaTabelas}{Sim}}{
\listoftables*
\cleardoublepage
}{}

\ifthenelse{\equal{\terSiglasAbrev}{Sim}}{
    \imprimirlistadesiglas
    \cleardoublepage
    \imprimirlistadesimbolos
    \cleardoublepage
 }{}

\ifthenelse{\equal{\terSumario}{Sim}}{
\tableofcontents*
}{}

\textual
\pagestyle{simple}

\parindent 1.50cm

\chapter[introducao]{INTRODUÇÃO}

No processo histórico, ao longo dos séculos, os instrumentos financeiros
passaram por profundas modificações e evoluções, assumindo papel determinante na
geração e acúmulo de riqueza para as famílias e nações. Nesse contexto surgiram
e se consolidaram as instituições bancárias, atuando essencialmente na segurança de depósitos e na oferta de crédito.

Na contrapartida destas atividades, os bancos são remunerados basicamente de
duas formas. A primeira delas é através de taxas para os serviços prestados. A
segunda se dá pelo resultado da diferença entre a taxa cobrada no oferecimento
de crédito e a taxa que remunera os recursos captados e utilizados para
empréstimo, caracterizando como o \emph{spread} bancário.

A medida que a oferta de crédito desponta como um fator fundamental para o
crescimento econômico de longo prazo, incentivando empreendimentos produtivos
contribuindo assim com a geração de emprego, renda e lucros, o \emph{spread} bancario
passa a ser um indicador estatégico para determinação do nível de desenvolvimento dos países e regiões.

A primeira via da importância do \emph{spread} bancário está relacionado com a
solidez do sistema financeiro. O nível deste indicador deve ser suficente p
ara garantir lucros atrativos para que as instituições mantenham suas atividades
e que novas tenham interesse em entrar no mercado, garantindo um setor forte,
gerando segurança e liquidez.

A segunda via remete a relação entre o custo de crédito e o nível de atividade
econômica. Onde segundo a teoria, um elevado nível de \emph{spread} bancário
desfavoreceria o crédito produtivo e consequentemente o nível de atividade
econômica impactando no crescimento e desenvolvimento do país ou região.

Tais premissas são sustentadas pelo Fundo Monetário Internacional e pelo Banco Mundial, que realizaram e incentivam estudos sobre o indicador a nível mundial.

\section{Contextualização}

\section{Justifictiva}

Foram utilizados os operadores boleanos em inglês (bank or banking) and
spread and brazil.

O lucro no setor bancário brasileiro é considerado muito elevado,
afetando principalmente o setor produtivo \cite{dantas:2012}. A américa latina
possui as maiores taxas de juros, bancos mais ineficientes e maiores
requerimentos de reserva, implicando maiores impactos no spread \cite{dantas:2012}

\textual
\pagestyle{simple}

\chapter{Referencial Teórico}

\section{Setor Bancário no Brasil}

Neste capítulo serão abordados os conceitos, características, composição e
evolução do setor bancário brasileiro, com objetivo de identificar variáveis
quantitativas e qualitativas relevantes para as análises dos componentes e
determinantes do \emph{spread} bancário.

Foi verificado o panorama das publicações de pesquisas relacionadas ao setor
bancário no brasil, através da plataforma Capes, entre os anos 2000 e 2020.
Foram utilizados os termos: Setor Bancário, Indústria Bancária, Mercado
Bancário, Estrutura Bancária e Brasil\footnote{Foram utilizados operadores
booleanos em inglês: banking(structure or market or sector or industry) and
brazil*.}, revisados por pares, remontando um total de 4.512 publicações,
indicando a relevância do tema.

Essa avaliação se torna relevante na concepção que o nível de desenvolvimento
do sistema financeiro guarda relação direta com grau de desenvolvimento
econômico do país ou região (ASSAF NETO (2014)

O setor bancário exerce papel socioeconômico fundamental, atuando na
intermediação financeira, promovendo a circulação do fluxo de crédito,
disponibilizando meios de pagamentos e opções para alocação de recursos MAFFILI
E SOUZA (2007)

O desenvolvimento do setor bancário pode ser influenciado por diversos fatores
endógenos --- relacionados com a gestão, tecnologia e eficiência de cada
instituição --- e exógenos --- envolvendo a regulação, conjuntura econômica e
social (ROVER et. al (2011).

Devido a importância de um sistema financeiro sólido no desenvolvimento
econômico de longo prazo, o lucro das instituições bancárias desperta constante
atenção em diversos países e regiões. Estas giram em torno dos riscos que
envolvem descontinuidade e insolvência (COUTO \emph{apud} \cite{dantas:2012}). De
acordo com Freitas e Khöler (2009) \emph{apud} \textcite{dantas:2012}, o Brasil
apresenta uma conjuntura bancária bem específica em comparação a outros países.

O setor bancário brasileiro é componente do Sistema Financeiro Nacional (SFN),
sob hierarquia normativa do Conselho Monetário Nacional (CMN) e supervisão do
Banco Central do Brasil (BACEN). As instituições que formam o setor bancário
assumem o papel de operadoras no mercado de crédito, atuando como
intermediadoras financeiras junto às pessoas físicas e jurídicas, podendo ser
de caráter público ou privado \cite{Lei:4595:1964}.

As modalidades de instituições no setor bancário brasileiro são os Bancos
Comerciais, Bancos de Investimentos, Bancos de Desenvolvimento, Bancos de
Câmbio, Bancos Múltiplos e Caixas Econômicas\footnote{Atualmente nessa
modalidade somente a Caixa Econômica Federal está em funcionamento}
\cite{Lei:4595:1964}.

\textcolor{red}{INSERIR PARÁGRAFO COM CARACTERÍSTICAS GERAIS DAS INSTITUIÇÕES BANCÁRIAS}

Os bancos comerciais são instituições financeiras de caráter público ou privado
constituídas na forma de sociedade anônima, atuando na intermediação de
recursos financeiros de curto e médio prazo para financiamento de atividades
comerciais, industriais, serviços, pessoas físicas e terceiros, realizando
captações através de depósitos à vista de livre movimento e depósitos à prazo
\cite{Res:2099:1994}.

A modalidade de Bancos de Investimento, as instituições financeiras devem ter
caráter privado, podendo operar participações temporárias em sociedades,
financiamentos produtivos para ativo fixo e capital de giro e gestão de
recursos de terceiros. Realizam captação de recursos por meio de depósitos a
prazo, repasses externos e internos e comercialização de cotas de fundos de
investimentos que administram \cite{Res:2624:1999}.

Na categoria de Bancos de Desenvolvimento, são autorizadas instituições
financeiras de caráter público, controladas por governos estaduais, com foco em
financiamento de atividades que promovam o desenvolvimento econômico regional
no médio e longo prazo, realizando operações passivas de depósitos a prazo,
recursos externos, endossos hipotecários e operações ativas de empréstimos e
financiamentos ao setor privado \cite{Res:394:1976}.

Os Bancos Múltiplos se caracterizam por instituições financeiras que, podem
assumir caráter público ou privado e, são autorizadas a realizar operações
ativas e passivas por meio de acumulação das carteiras comercial, investimento,
desenvolvimento, crédito imobiliário, arrendamento mercantil e crédito, financiamento e investimento \cite{Res:2099:1994}.

Em sua composição os Bancos Múltiplos devem assumir no mínimo duas carteiras e,
de forma obrigatória, uma delas, deve ser a comercial ou a de investimento. As
que optarem por carteira comercial podem realizar captação via depósito à vista. Somente os Bancos Públicos podem acumular a carteira de desenvolvimento
\cite{Res:2099:1994}.

No segmento de Bancos de Câmbio, as instituições financeiras possuem
autorização para realizar operações compra e venda de crédito cambial. Entre as
operações de crédito estão o financiamento para exportadores e importadores e
antecipação mediante contratos cambiais. Podem receber depósitos em contas com
movimentação restrita e sem remuneração exclusiva para as operações cambiais
\cite{Res:3426:2006}.

A Caixa Econômica Federal (CEF), fundada em 1861, e regulamentada pelo
Decreto-Lei nº 759 de 1969 é uma empresa pública subordinada ao Ministério da
Economia, com operações similares a de um Banco Comercial, priorizando projetos
e programas relacionados a área social e infraestrutura \cite{DL:759:1969}.

A CEF atua com operações de crédito ao consumidor, para financiamento de bens
de consumo duráveis, operações de garantia de penhor industrial e caução de
títulos. Detém o monopólio sobre o penhor de bens pessoais e venda de bilhetes
de loteria. É integrante do Sistema Financeiro da Habitação (SFH) e Sistema
Brasileiro de Poupança e Empréstimo (SBPE), além da detenção centralizado do
recolhimento e aplicação dos recursos do FGTS \cite{DL:759:1969}.

O setor bancário brasileiro passou por significativas modificações em sua
estrutura no final da década de 1980 e ao longo da década de 1990. Estas
modificações ocorreram em grande parte como reflexo às mudanças internacionais
e ao processo de abertura comercial e financeira que se iniciou no Brasil
\cite{camargo:2009}.

Na \autoref{tab:banks} é possível verificar a concentração --- levando em
consideração a quantidade de instituições --- do setor bancário brasileiro na
categoria de bancos múltiplos, com 76\%,3 de participação, onde apenas 11,5\% das
instituições bancárias operam exclusivamente com carteira comercial e 6,3\%
exclusivamente com investimento.

\begin{table}
\caption{Composição do setor bancário brasileiro por segmento em dezembro de 2019}
\begingroup\fontsize{10}{12}\selectfont

\begin{tabu} to \linewidth {>{\raggedright\arraybackslash}p{6cm}>{\raggedright}X>{\raggedright}X>{\raggedleft}X>{\raggedright}X}
\toprule
Segmento & Sigla & Ano & Quantidade & Participação\\
\midrule
\cellcolor{gray!6}{Banco Múltiplo} & \cellcolor{gray!6}{BM} & \cellcolor{gray!6}{2019} & \cellcolor{gray!6}{132} & \cellcolor{gray!6}{76.30\%}\\
Banco Comercial & BC & 2019 & 20 & 11.56\%\\
\cellcolor{gray!6}{Banco de Investimento} & \cellcolor{gray!6}{BI} & \cellcolor{gray!6}{2019} & \cellcolor{gray!6}{11} & \cellcolor{gray!6}{6.36\%}\\
Banco de Câmbio & B Camb & 2019 & 5 & 2.89\%\\
\cellcolor{gray!6}{Banco de Desenvolvimento} & \cellcolor{gray!6}{BD} & \cellcolor{gray!6}{2019} & \cellcolor{gray!6}{4} & \cellcolor{gray!6}{2.31\%}\\
\addlinespace
Caixas Econômicas & CE & 2019 & 1 & 0.58\%\\
\bottomrule
\end{tabu}
\endgroup{}
\label{tab:banks}
\fonte{Desenvolvido com dados do Banco Central}
\end{table}

Entre as principais mudanças iniciadas na década de 1980 está a reforma
bancária ocorrida em 1998, através da Resolução nº 1.524 \cite{Res:1524:1988},
que instituiu diversas medidas de desregulamentação, entre elas a extinção da
necessidade de carta-patente para constituição de Bancos Múltiplos.

Mesmo com as limitações da Constituição de 1988 \cite{constituicao:1988} para
instalação de bancos estrangeiros, não houveram restrições para que ocorresse
aumento na participação de capital estrangeiro em bancos nacionais
\cite{camargo:2009}.

\begin{figure}
\captionof{figure}{Evolução do setor bancário brasileiro por segmento}

\begin{center}\includegraphics{12-exportedfigures/bank evolution-1} \end{center}
\label{fig:segmento}
\fonte{Desenvolvido a partir de dados do Banco Central}
\end{figure}

\textcolor{red}{ABORDAR SOBRE O ÍNDICE HHI (CONCENTRAÇÃO)}

A \autoref{fig:segmento} demonstra a evolução número de instituições bancárias
por segmento entre 1978 à 2019, podendo ser visualizada uma mudança na
composição da estrutura, com significativo aumento de instituições aderindo
modalidades de múltiplas carteiras \footnote{As primeiras instituições com
carteira múltipla começaram a operar no ano de 1988} e redução de instituições que operam exclusivamente com carteira comercial e exclusivamente com carteira
de investimento.

\begin{table}
\caption{Composição por tipo de iniciativa no setor bancário brasileiro — Dezembro 2019}
\begingroup\fontsize{10}{12}\selectfont

\begin{tabu} to \linewidth {>{\raggedright\arraybackslash}p{6cm}>{\raggedright\arraybackslash}p{6cm}}
\toprule
Tipo & Participação\\
\midrule
\cellcolor{gray!6}{Privado} & \cellcolor{gray!6}{93\%}\\
Público & 7\%\\
\bottomrule
\end{tabu}
\endgroup{}
\label{tab:iniciativa}
\fonte{Desenvolvido pelo autor, com dados do Banco Central}
\end{table}

Alguns dos efeitos da abertura comercial-financeira e das modificações na
estrutura bancária provenientes das medidas governamentais foram o aumento da
participação de instituições estrangeiras no país e, um consistente processo de
fusões e aquisições, de ambas as origens de capital, que resultou em
considerável elevação do grau de concentração \cite{camargo:2009}.

\begin{figure}
\captionof{figure}{Evolução da quantidade de instituições no setor bancário brasileiro}

\begin{center}\includegraphics{12-exportedfigures/concetration-1} \end{center}
\label{fig:concentracao}
\fonte{Desenvolvido pelo autor, com dados do Banco Central}
\end{figure}

A observação sobre o aumento da concentração bancária no Brasil realizada por
\textcite{camargo:2009} pode ser visualizada na \autoref{fig:concentracao}.
Entre as metades das décadas de 1980 e 1990, com redução da concentração,
levando em consideração o número de instituições. Esse cenário passou se
inverter a partir de 1994, chegando em 2019 a um nível aproximado ao observado
no início da década de 1980.

De acordo com Strachman e Vasconcelos \emph{apud} \textcite{camargo:2009}, o aumento
da concentração bancária pode ser prejudicial ao crescimento econômico, uma vez
que, com maior participação de mercado, as instituições bancárias acabam por
obter a prerrogativa de determinar seus preços, comportamento este observado em
\textcite{klein:1971}.

Segundo \textcite{camargo:2009} e \textcite{dantas:2012} por outra perspectiva,
o ganho de escala, onde o cenário de aumento do tamanho das instituições, das
operações de crédito e redução de custos operacionais atua melhorando a
remuneração dos depósitos podendo atuar na redução dos juros finais pagos pelos
clientes.

Outra possível tendência para a concentração bancária seria a redução do risco
das operações, implicando em redução de custos, obtida por meio expansão
geográfica, setorial e de produtos financeiros. Porém os possíveis efeitos da
concentração dependem de uma série de condições, principalmente em torno da
eficiência e do nível de concorrência no mercado \cite{camargo:2009}.

\begin{table}
\caption{Setor bancário brasileiro por origem de capital — Dezembro de 2019}
\begingroup\fontsize{10}{12}\selectfont

\begin{tabu} to \linewidth {>{\raggedright}X>{\raggedleft}X>{\raggedright}X}
\toprule
Capital & Quantidade & Participação\\
\midrule
\cellcolor{gray!6}{Nacionais} & \cellcolor{gray!6}{66} & \cellcolor{gray!6}{43.1\%}\\
Controle Estrangeiro & 60 & 39.2\%\\
\cellcolor{gray!6}{Nacionais com Participação Estrangeira} & \cellcolor{gray!6}{12} & \cellcolor{gray!6}{7.8\%}\\
Públicos & 10 & 6.5\%\\
\cellcolor{gray!6}{Estrangeiros} & \cellcolor{gray!6}{5} & \cellcolor{gray!6}{3.3\%}\\
\bottomrule
\end{tabu}
\endgroup{}
\label{tab:origemcapital}
\fonte{Desenvolvida pelo autor, com dados do Banco Central}
\end{table}

\begin{figure}
\captionof{figure}{Evolução de origem de capital das instituições bancárias no Brasil}

\begin{center}\includegraphics{12-exportedfigures/capital graphic-1} \end{center}
\label{fig:ev.capital}
\fonte{Desenvolvido pelo autor, com dados do Banco Central}
\end{figure}

O aumento da participação estrangeira no setor bancário brasileiro durante a
década de 1990, evidenciado por \textcite{camargo:2009} pode ser observado na
\autoref{fig:ev.capital}. Esse aumento ocorreu principalmente através do
controle acionário, com elevação acentuada na segunda metade da década de 1990
até o início da década de 2000. Ocorrendo redução em instituições
nacionais, estrangeiras e nacionais com participação estrangeira.

Durante este período, a inclinação para aplicação massiva em títulos públicos
se dava diante a manutenção de elevadas taxas de juros, tornando o crédito para
empreendimentos privados de elevado risco, e consequentemente elevando
substancialmente o \emph{spread} bancário e reduzindo a oferta de crédito
\cite{camargo:2009}.

A expectativa com a entrada de instituições estrangeiras era que houvesse
elevação da concorrência e, consequentemente, redução no \emph{spread} bancário,
aumento da concessão de crédito, melhoria da qualidade e diversificação dos
produtos financeiros, avanços em tecnologias, ou seja, uma elevação na
eficiência do setor \cite{camargo:2009}.

Porém, o que se observou foi a adoção de postura conservadora por partes dos
bancos estrangeiros, com estratégia de ativos inclinada para negociação de
títulos públicos, e passivos direcionados para a captação de recursos advindos
de grupos de rendas média e alta, com exceção dos bancos públicos que
concentraram em operações de crédito \cite{camargo:2009}.

Segundo Singh \emph{apud} \textcite{leal:2006}, durante a década de 1990 o \emph{spread}
bancário no Brasil esteve superior a 50\%a.a., enquanto na América Latina esteve
entre 10\% a 15\% a.a. A relação crédito/PIB em 2003 no Brasil era de 23\%,
considerado muito baixo em comparação Chile com 68,5\%, Uruguai com 64,3\%, Estados Unidos com 60,8\%, Japão com 64,3\%, Coréia com 98,9\% e Europa com 140,6\% \cite{camargo:2009, leal:2006}.

\begin{figure}
\captionof{figure}{Evolução da relação Crédito/PIB no Brasil}

\begin{center}\includegraphics{12-exportedfigures/credit gdp-1} \end{center}
\label{fig:credgdp}
\fonte{Desenvolvido pelo autor, com dados o Banco Central}
\end{figure}

a \autoref{fig:credgdp} demonstra o comportamento da relação crédito/PIB no
Brasil, que entre a segunda metade da década de 1990 até a meados da primeira
metade da década de 2000 sofreu significativa queda, ficando abaixo dos 25\%.
Após esse período a oferta de crédito sofreu uma expansão exponencial atingindo
patamares acima de 50\% do PIB.

Durante o período citado, foi observado no setor bancário brasileiro os maiores
níveis de \emph{spread} praticados no mundo, associado a um quadro econômico
instabilidades e baixos crescimento e desenvolvimento. Esse cenário encontra
embasamento em estudos teóricos e empíricos que demonstram que um sistema
financeiro desenvolvido favorece o crescimento e desenvolvimento econômico
\cite{levine:1997, matos:2003}.

\begin{figure}
\captionof{figure}{Evolução anual do saldo carteira de crédito}

\begin{center}\includegraphics{12-exportedfigures/balance credit-1} \end{center}
\label{fig:saldocredito}
\fonte{Elaborado com dados do Banco Central}
\end{figure}

A \autoref{fig:saldocredito} demonstra a evolução do saldo da carteira de
crédito anual em termos correntes entre 1990 e 2020, podendo ser visualizada
uma expansão exponencial de crédito a partir do início da década de 2000, com
leve recuo na até metade da década de 2010 e retoma ultrapassando máxima
anterior.

Diante o levantamento, o setor bancário brasileiro durante o período avaliado
passou por diversas transformações em sua estrutura no que tange a concentração
de mercado, aumento da participação de capital estrangeiro por meio de controle
acionário, redução da participação pública.

Em relação aos indicadores foi verificado que entre a década de 1980 até metade
da década de 1990, no cenário hiperinflacionário, mesmo com redução da
concentração bancária, os indicadores de eficiência de intermediação
financeiras como o \emph{spread} bancário e a relação crédito/PIB estavam em níveis
considerados ineficientes e muito destoantes em comparação a outros países e
regiões.

A partir de 1995 se observou mudanças significativas no setor bancário, com
nova concentração, redução de instituições nacionais devido o controle
acionário por capital estrangeiro, e expressiva redução no \emph{spread} bancário e
a partir de 2004 uma mudança significativa na relação crédito/PIB.

Este capítulo levantou informações amplas sobre o setor bancário brasileiro, e
identificou como variáveis o nível de concentração, tipo de iniciativa, origem
do capital, taxa de juros, \emph{spread} bancário, saldo da carteira de crédito de
recursos livres e destinados, PIB\ldots{} No próximo capítulo serão levantados
conceitos, definições e estudos sobre a evolução, decomposição e determinantes
do \emph{spread} bancário.

\textcolor{red}{INSERIR BREVE CONTEXTO E DADOS SOBRE BASE MONETÁRIA}

\textual
\pagestyle{simple}

\section{Spread Bancário}

Este capítulo irá tratar sobre os principais aspectos e características do
\emph{spread} bancário. Na primeira parte serão abordados conceitos e definições
gerais. Na segunda parte as características amplas do mercado Brasileiro. Na
terceira parte sobre os estudos empíricos realizados no Brasil. O foco é
identificar elementos que possam contribuir com o objeto deste estudo.

\section{Conceitos e Definições}

Por definição o \emph{spread} bancário é obtido através da subtração percentual,
entre a taxa de aplicação incidente nas operações de crédito, e a taxa de
captação que remunera as aplicações financeiras, se configurando como a
diferença entre a composição dos custos destas operações \cite{BCB:2000}.

\[
Spread = Taxa de Aplicação - Taxa de Captação
\]

O \emph{spread} bancário representa uma medida que sinaliza o desempenho dos bancos
\cite{levine:1997}. É considerado um indicador de eficiência da economia, no
sentido de favorecer o crédito e a atividade econômica. Em níveis elevados pode
desfavorecer o crédito destinado para produção e consumo produtivos e estar
associado com o fraco desempenho econômico \cite{WB:2005}.

Os estudos em torno do \emph{spread} bancário ocorrem em três óticas: evolução,
estrutura e determinantes (SOUZA (2014). Em Dick \emph{apud} \cite{leal:2006} é
destacada a importância de distinguir a abordagem em torno da estrutura e
determinante do \emph{spread} bancário, no sentido de complementariedade. O diagrama
na \autoref{fig:diagram} ilustra as óticas de estudo do \emph{spread} bancário.

\begin{figure}
\captionof{figure}{Diagrama de ilustração das vertentes de pesquisa do \emph{spread}}

\begin{center}\includegraphics{12-exportedfigures/diagram.spread-1} \end{center}
\label{fig:diagram}
\fonte{Desenvolvido com base nas fontes citadas}
\end{figure}

A abordagem em torno da evolução visa analisar o comportamento ao longo do
tempo, através de análises quantitativas e qualitativas, enquanto a ótica da
estrutura busca identificar e analisar os componentes de resultado envolvendo
receitas, despesas e provisões. Na abordagem sobre os determinantes é
vislumbrado identificar as variáveis que explicam as variações do indicador ao
longo dos períodos \cite{leal:2006}.

Vem se tornando relevantes os estudos em torno da decomposição do \emph{spread}
bancário. Entre os componentes explícitos estão a inadimplência, despesas
administrativas, impostos diretos e indiretos e margem de lucro dos bancos
conforme ilustrado abaixo \cite{BCB:2000}.

\[
Sprd=f(Ind, DA, II, ML, CP)
\]

\begin{itemize}
\tightlist
\item
  Sprd = \emph{Spread}
\item
  Ind = Inadimplência
\item
  DA = Despesas Administrativas
\item
  II = Impostos Indiretos
\item
  ML = Margem de Lucro
\item
  CP = Custo de Captação
\end{itemize}

Esta configuração dos componentes, contemplando a margem de lucro, despesas e
riscos envolvidos nas operações de crédito vem desmistificar a comum abordagem
do \emph{spread} como o rendimento auferido pelos bancos \cite{costa;nakane:2004}
Souza (2007) \emph{apud} \cite{dantas:2012}. Desta forma se configurando como a
diferença entre o custos operacionais na ótica de precificação, que após
descontados, remontam o lucro do banco \cite{BCB:2016}.

Além da avaliação de seus componentes, o \emph{spread} pode ser analisado
conjuntamente por três características: enquanto a abrangência da amostra,
conteúdo e origem da informação \cite{leal:2006}.

A abrangência da amostra consiste nas especificidades das operações de crédito
das instituições e seu nível de agregação e granularidade
\cite{costa;nakane:2004}. Uma análise agregada dessa característica pode ser
dificultada pela existência de heterogeneidade do setor, ressaltando a
importância de realizar análises do \emph{spread} bancário em diferentes
características e óticas \cite{block:2000}.

A abordagem em torno do conteúdo está relacionada com os subcomponentes que
envolvem a receita e as despesas das intermediações financeiras, podendo
englobar, ou não, as tarifas e comissões sobre as taxas de captações e
aplicação \cite{block:2000}.

A origem da informação é analisada em dois cenários: \emph{ex-ante} e \emph{ex-post}
\cite{kunt:1999, levine:1997}. A perspectiva \emph{ex-ante} se refere as
expectativas das instituições bancárias em relação ao mercado de crédito e os
riscos envolvidos, obtido por método de precificação envolvendo as taxas de
captação e empréstimo \cite{durigan:2018, leal:2006, dantas:2012}.

O \emph{spread} \emph{ex-ante}, por se tratar de um indicador de planejamento, refletindo
as expectativas das instituições bancárias em relação ao mercado, finda
demonstrando-se mais volátil, não representando as taxas efetivas realizadas.
As informações \emph{ex-ante} são repassadas ao Banco Central que as divulgam
\cite{durigan:2018, leal:2006, dantas:2012}.

No \emph{spread ex-post} as margens são obtidas mediante a apuração dos resultados
contábeis, através dos demonstrativos, considerando as receitas e custos
efetivos, implicando nas taxas de intermediação e carteira realizadas pelas
instituições financeiras \cite{kunt:1999, durigan:2018}. Nesse sentido, em
termos médios, as taxas \emph{ex-post} se demonstram mais estáveis \cite{leal:2006, dantas:2012}.

Reduções no \emph{spread} \emph{ex-post} não necessariamente significam aumento da
eficiência da intermediação financeira, pois podem estar associadas a uma
redução da inadimplência \cite{kunt:1999}. Como observado em
\textcite{klein:1971} e \textcite{ho-saunders:1981} o \emph{spread} bancário é
determinado de acordo com as características e os riscos envolvidos nas
intermediações financeiras inerentes em cada estrutura de mercado.

\section{Spread Bancário no Brasil}

No Brasil, a taxa de aplicação para crédito de recursos livres é pactuado entre
instituição e tomador. Somente as operações de crédito envolvendo recursos
direcionados são sujeitas à limites, não podendo exceder 12\%a.a. mais a taxa
referencial de juros \cite{BCB:2016}.

No mercado bancário Brasileiro, o modelo consolidado de mensuração do \emph{spread},
conforme \autoref{tab:spread.tb}, leva em consideração o saldo médio de capital
emprestado, e a diferença entre as receitas de aplicação e despesas de
captação, ocorrendo a classificação em \emph{spread} bruto, direto e líquido
\cite{fipecafi:2005}

\begin{table}[b]
 \centering
   \caption{Esquema de obtenção do \emph{spread} mais adotado no mercado} 
    \label{tab:spread.tb}
     \begin{tabular}{l|c|c|c}
      \hline
                                           &   PJ   &   PF    & Total \\
       \hline
       Saldo Médio do Capital Emprestado   & 100.00 & 100.00  & 100.00 \\
       A — Receita de Aplicação Financeira & 9.4    & 16.5    & 12,7   \\
       B — Despesas de Captação            & (4.8)  & (4.9)   & (4.8)  \\   
       Spread Bruto                        & 4.6    & 11.6    & 7.9    \\
       Spread Direto                       & 3.2    & 7.6     & 5.3    \\
       Spread Líquido                      & 0.5    & 1.6     & 1.0    \\
       \hline
       \end{tabular}
\fonte{in \cite{fipecafi:2005}}
\end{table}

O Banco Central, em 1999, iniciou uma série de estudos e medidas com objetivo
de reduzir a taxa de juros e o \emph{spread} realizados no setor bancário
Brasileiro, atuando na identificação e ajustes em variáveis econômicas
influentes. Entre as primeiras medidas estavam a redução da taxa de compulsório
para depósitos à vista e até a extinção para depósitos à prazo, redução do IOF
e a redução da Selic \cite{BCB:2000}.

\begin{figure}
\captionof{figure}{Evolução do \emph{spread} bancário Brasileiro até 2011}

\begin{center}\includegraphics{12-exportedfigures/average spread-1} \end{center}
\label{fig:spread2012}
\fonte{Desenvolvido a partir de dados do Banco Central}
\end{figure}

A \autoref{fig:spread2012} mostra a evolução do \emph{spread} bancário Brasileiro
médio entre os anos de 1994 e 2012, chegando a atingir 146.44\%, com
significativa queda ao longo desse período, atingindo 24.69\% no final
do período. Esta série foi descontinuada em 2012, passando a ser utilizada nova
metodologia de cálculo.

O Banco Central, até 2007 utilizava metodologia para avaliação do spread
bancário contemplando somente os recursos livres, o que não vinha a
proporcionar uma avaliação mais aprofundada. Em 2008 houve uma modificação na
metodologia de decomposição do \emph{spread}, alterando o cálculo do custo médio de
captação e detalhando classificações do crédito \cite{dantas:2012}

Para o custo médio de captação passou a se utilizar a taxa média ponderada
entre as taxas dos depósitos à prazo (CDB), em caderneta de poupança e à vista,
a participação dos custos efetivos dos recolhimentos compulsórios em detrimento
do custo de oportunidade \cite{dantas:2012}

O BACEN mantém atualmente duas séries para o indicador: spread médio das
operações de crédito (MOC) e Spread do Indicador de Custo de Crédito (ICC). As
séries são disponibilizadas em termos totais e nas subdivisões de tipo de
recursos, tipo de crédito e tomador.

Estas séries estatísticas representam estimativas baseadas nas informações
repassadas pelas instituições bancárias das taxas de juros das operações de
crédito e indicadores do mercado financeiro do custo médio do dinheiro para o
custo médio de captação \cite{BCB:2016}.

A série do Spread médio das operações de crédito é calculada a partir da
diferença entre a taxa média de juros de novas operações de crédito no SFN e o
custo de captação referencial médio de operações de crédito livre, direcionado
e não rotativo podendo ser observados por tomador.

\begin{figure}
\captionof{figure}{Evolução do Spread médio das operações de crédito}

\begin{center}\includegraphics{12-exportedfigures/spread 2019 moc-1} \end{center}
\label{fig:spreadmoc}
\fonte{Desenvolvido a partir de dados do Banco Central do Brasil — Departamento de Estatísticas}
\end{figure}

A \autoref{fig:spreadmoc} mostra a visualização da evolução mensal do spread
médio das novas operações de crédito contratadas entre janeiro de 2013 e julho
de 2020. No período entre 2014 e 2017 se visualiza uma elevação de 10 p.p no
spread total, recuando 8 p.p a patamar próximo ao início do período. É possível
notar a grande disparidade entre os spread de recursos livres e direcionados.

A série do Spread do ICC, considera a diferença entre o Índice de Custo de
Crédito --- equivalente ao custo médio de juros das operações ativas da carteira
do SFN --- e o custo de captação médio ponderado, levando em consideração
operações de crédito livre, direcionado e não rotativo, dividido em pessoa
física e jurídica.

\begin{figure}
\captionof{figure}{Evolução do Spread do Índice do Custo de Crédito}

\begin{center}\includegraphics{12-exportedfigures/spread 2019 icc-1} \end{center}
\label{fig:spreadicc}
\fonte{Desenvolvido a partir de dados do Banco Central do Brasil - Departamento de Estatísticas}
\end{figure}

Na \autoref{fig:spreadicc} pode ser visualizada a evolução do spread do ICC,
entre janeiro de 2013 e julho de 2020 com expressiva elevação entre 2014 e
2017, passando a decair até retormar a patamares similares ao início do
período. Também pode ser notado a significativa diferença entre os \emph{spreads} de
recursos livres e direcionados.

O Indicador Custo de Crédito (ICC) consiste no custo médio de todas as
operações de crédito abertas --- independentes do período em que foram
contratadas --- que compõem a carteira de empréstimos, financiamentos e
arrendamento mercantil das instituições do Sistema Financeiro Nacional (SFN) \cite{BCB:2000}.

\begin{figure}
\captionof{figure}{Evolução do Indicador de Custo de Crédito (ICC)}

\begin{center}\includegraphics{12-exportedfigures/ICC-1} \end{center}
\label{fig:evicc}
\fonte{Desenvolvido a partir de dados do Banco Central do Brasil — Departamento de Estatísticas}
\end{figure}

A \autoref{fig:evicc} traz a visualização da evolução do Índice de Custo de
Crédito entre janeiro de 2013 e agosto de 2020, com máxima de 22.98\% em
2017, com queda significativa a partir de 2020, chegando a atingir 17.89\%
em agosto de 2020.

\section{Estudos anteriores}

Na literatura acadêmica não existe uma teoria formalizada acerca do \emph{spread}
bancário \cite{timotio:2018}. Sendo verificados estudos empíricos que visam
classificar, analisar e identificar variáveis micro e macroeconômicas
influentes nesse indicador em diversas perspectivas.

A grande maioria dos estudos realizados no Brasil utilizam as medidas de
\emph{spread} bancário divulgadas pelo Banco Central, que remetem a uma perspectiva
\emph{ex-ante}, registrando as taxas planejadas na fase de concessão de crédito. E
para as variáveis explicativa a grande maioria utiliza indicadores
macroeconômicos \cite{dantas:2012}

No ano de 1994, \textcite{aronovich:1994} realizou estudo econométrico para
verificar a influência da inflação e nível de atividade econômica no \emph{spread}
bancário ex-ante, encontrando relação direta do \emph{spread} com a inflação e
indireta com o nível de atividade econômica.

Em estudo dos determinantes macroeconômicos do \emph{spread} bancário ex-ante,
\textcite{oreiro-2006} utilizou regressão múltipla para identificar as
variáveis influentes (modelo abaixo). O estudo chegou ao resultado que alta
volatilidade e as taxas da Selic são um dos principais determinantes desse
indicador no setor bancário Brasileiro, identificando também a significância do
nível de atividade industrial.

\[
ln spread = \beta_0 trend + \beta_1 ln selic + \beta_2 ln adm + \beta_3 ln risk + \beta_4 ln imp + \beta_5 ln comp
\]

\begin{itemize}
\tightlist
\item
  \(\beta_i\) (i= 0,\ldots{}, 5) = parâmetros estimados;
\item
  trend = tendência determinista que controla outras variáveis;
\item
  selic = taxa Selic;
\item
  adm = despesa administrativas;
\item
  risk = proxy para o risco de crédito (spread do C-Bond sobre o rendimento dos títulos do Tesouro Americano de mesma maturidade;
\item
  imp são impostos indiretos;
\item
  comp = compulsório incidente sobre os depósitos à vista.
\end{itemize}

Em análise dos determinantes do \emph{spread} bancário ex-post,
\textcite{dantas:2012} utilizou variáveis explanatórias microeconômicas de cada
instituição, por meio de dados em painel dinâmico, entre janeiro de 2000 e
outubro de 2009, encontrando níveis significativos e diretos com o risco de
crédito, grau de concentração e nível de atividade econômica, e indireta com a
participação da instituição no mercado, não encontrando níveis significativos
com origem de capital e tipo de organismo.

Outra observação em \textcite{dantas:2012} foi a forte relação do \emph{spread}
\emph{ex-post} no momento atual com o momento anterior imediato, e que as
instituições tendem a cobrar maiores taxas, quando maior o nível de
concentração do mercado, não encontrando significância da Selic na determinação
deste indicador.

Em \textcite{almeida:2013} foi desenvolvido modelo de dados macroeconômicos e
microeconômicos em painel, de 64 instituições bancárias para avaliação de
determinantes do \emph{spread} \emph{ex-post} no Brasil entre o primeiro trimestre de
2001 e o segundo trimestre de 2012, encontrando como relevantes
as despesas administrativas, receita de serviços, índice de cobertura, PIB e o
grau de concentração.

Em \textcite{durigan:2018} foi realizada análise dos fatores macroeconômicos e
indicadores industriais que influenciam o \emph{spread} bancário ex-ante, através de
análise de regressão linear multivariada utilizando 18 variáveis em quatro
modelos. Chegando a conclusão que o aumento da atividade industrial, a redução
do desemprego e o consumo atuam na diminuição do \emph{spread} bancário.

Os modelos desenvolvidos por \textcite{durigan:2018} demonstraram que há uma
relação significativa e direta entre \emph{spread} e: inadimplência, IPIs (bens de
capital, intermediários, semiduráveis, não duráveis e consumo duráveis), Selic,
PIB, desemprego e o EMBI+ (medida de taxa de risco-país). As relações indiretas
com o \emph{spread} foram encontradas: no IPI de bens de consumo e geral, IPCA,
saldo da carteira de crédito e índice de vendas no varejo.

O estudo de \textcite{timotio:2018} teve foco em abordagem microeconômica, ao
buscar identificar a influência das variações de indicadores
financeiros-contábeis no \emph{spread} em 26 instituições bancárias,
através de regressão em dados em painel. Encontrando relações significativas
diretas com a alavancagem financeira, retorno sobre o patrimônio líquido,
EBITDA, Ativo Total e eficiência.

No modelo de \textcite{timotio:2018} foi encontrada relação significativa e
indireta do \emph{spread} com a participação de capital de terceiros e, não
identificada relação significativa com a composição do endividamento, retorno
sobre ativos e a liquidez corrente.

De acordo com \textcite{durigan:2018, dantas:2012}, existem poucos estudos
inclinados para os determinantes do \emph{spread} \emph{ex-post} no Brasil, onde
identificaram o estudos de Guimarães (2002). Foram identificados ainda os
estudos acerca do \emph{spread} ex-pots de Fipecafi (2004) \emph{apud}
\textcite{dantas:2012} e Matias (2006) \emph{apud} \textcite{leal:2006}

Em \textcite{fipecafi:2005} foi realizado estudo de apuração de resultados,
ex-post, baseado em demonstrações contábeis entre o 1º semestre de 2005 de
instituições que representavam 75,8\% do ativo total e 76\% do total de crédito.
Chegando a um resultado médio de \emph{spread} bruto de 7,6\% para pessoa física e
3,2\% para pessoa jurídica, e \emph{spread} líquido de 1,6\% para pessoa física e 0,5\%
para pessoa jurídica.

A \autoref{tab:exantea} e a \autoref{tab:exanteb} trazem o resumo dos
principais estudos empíricos sobre \emph{spread} bancário ex-ante no Brasil, com
resultados obtidos através de modelagem econométrica com utilização de
regressão, tomando variáveis micro e macroeconômicas como explanatórias e
demonstrando a relação com o spread ex-ante.

\begin{table}
\caption{Resumo de estudos sobre o \emph{spread ex-ante} no Brasil — Parte 1}
\begin{table}[H]
\centering\begingroup\fontsize{10}{12}\selectfont

\begin{tabular}[t]{>{\raggedright\arraybackslash}p{4cm}>{\raggedright\arraybackslash}p{2cm}>{\raggedright\arraybackslash}p{2cm}>{\raggedright\arraybackslash}p{2cm}>{\raggedright\arraybackslash}p{2cm}}
\toprule
Variável & KOYAMA e NAKANE (2001a e 2001b) & AFANASIEFF, LHAGER e NAKANE (2001) & AFANASIEFF, LHAGER e NAKANE (2002) & BIGNOTTO e RODRIGUES (2006)\\
\midrule
\cellcolor{gray!6}{Custos Administrativos} & \cellcolor{gray!6}{+} & \cellcolor{gray!6}{+} & \cellcolor{gray!6}{+} & \cellcolor{gray!6}{+}\\
\textbf{IGP} & + & + & - & \\
\cellcolor{gray!6}{Impostos Indiretos} & \cellcolor{gray!6}{+} & \cellcolor{gray!6}{+} & \cellcolor{gray!6}{+} & \cellcolor{gray!6}{}\\
\textbf{Requerimento de Reserva} & + &  &  & \\
\cellcolor{gray!6}{Selic} & \cellcolor{gray!6}{+} & \cellcolor{gray!6}{+} & \cellcolor{gray!6}{+} & \cellcolor{gray!6}{+}\\
\addlinespace
\textbf{Spread Over Treasury} & + &  & + & \\
\cellcolor{gray!6}{Produto Industrial} & \cellcolor{gray!6}{-} & \cellcolor{gray!6}{} & \cellcolor{gray!6}{} & \cellcolor{gray!6}{}\\
\textbf{Ativo Total} &  &  &  & +\\
\cellcolor{gray!6}{Bancos Estrangeiros} & \cellcolor{gray!6}{} & \cellcolor{gray!6}{} & \cellcolor{gray!6}{-} & \cellcolor{gray!6}{}\\
\textbf{Captação sem juros} &  & + & + & \\
\addlinespace
\cellcolor{gray!6}{Compulsório} & \cellcolor{gray!6}{} & \cellcolor{gray!6}{} & \cellcolor{gray!6}{} & \cellcolor{gray!6}{+}\\
\textbf{Crescimento PIB Industrial} &  & - & + & \\
\cellcolor{gray!6}{IPCA} & \cellcolor{gray!6}{} & \cellcolor{gray!6}{} & \cellcolor{gray!6}{} & \cellcolor{gray!6}{-}\\
\textbf{Liquidez} &  &  &  & +\\
\cellcolor{gray!6}{Market Share} & \cellcolor{gray!6}{} & \cellcolor{gray!6}{} & \cellcolor{gray!6}{} & \cellcolor{gray!6}{-}\\
\addlinespace
\textbf{Receita Serviços} &  & + & + & +\\
\cellcolor{gray!6}{Risco Crédito} & \cellcolor{gray!6}{} & \cellcolor{gray!6}{} & \cellcolor{gray!6}{} & \cellcolor{gray!6}{+}\\
\textbf{Risco Juros} &  &  &  & +\\
\cellcolor{gray!6}{Volatilidade da Selic} & \cellcolor{gray!6}{} & \cellcolor{gray!6}{-} & \cellcolor{gray!6}{} & \cellcolor{gray!6}{}\\
\bottomrule
\end{tabular}
\endgroup{}
\end{table}
\label{tab:exantea}
\fonte{Desenvolvido a partir das fontes citadas}
\end{table}

Entre os estudos da \autoref{tab:exantea} e \autoref{tab:exanteb} que avaliaram
a Selic e as despesas administrativas, há um consenso que estas variáveis
possuem uma relação de determinação direta com o \emph{spread ex-ante}. Em três
estudos que avaliaram impostos indiretos e receita de serviços foi encontrada
relação direta com o \emph{spread ex-ante}.

\begin{table}
\caption{Resumo de estudos sobre o \emph{spread ex-ante} no Brasil — Parte 2}
\begin{table}[H]
\centering\begingroup\fontsize{10}{12}\selectfont

\begin{tabular}[t]{>{\raggedright\arraybackslash}p{4cm}>{\raggedright\arraybackslash}p{3cm}>{\raggedright\arraybackslash}p{3cm}>{\raggedright\arraybackslash}p{3cm}}
\toprule
Variável & OREIRO et al. (2006) & DURIGAN (2018) & ARONOVICH (1994)\\
\midrule
\cellcolor{gray!6}{Selic} & \cellcolor{gray!6}{+} & \cellcolor{gray!6}{+} & \cellcolor{gray!6}{}\\
\textbf{Produto Industrial} & + &  & \\
\cellcolor{gray!6}{Atividade Econômica} & \cellcolor{gray!6}{} & \cellcolor{gray!6}{} & \cellcolor{gray!6}{-}\\
\textbf{Desemprego} &  & + & \\
\cellcolor{gray!6}{EMBI} & \cellcolor{gray!6}{} & \cellcolor{gray!6}{+} & \cellcolor{gray!6}{}\\
\addlinespace
\textbf{Inadimplência} &  & + & \\
\cellcolor{gray!6}{Índice Volume Vendas Varejo} & \cellcolor{gray!6}{} & \cellcolor{gray!6}{-} & \cellcolor{gray!6}{}\\
\textbf{IPCA} &  & - & +\\
\cellcolor{gray!6}{IPI bcd} & \cellcolor{gray!6}{} & \cellcolor{gray!6}{+} & \cellcolor{gray!6}{}\\
\textbf{IPI Bens de Capital} &  & + & \\
\addlinespace
\cellcolor{gray!6}{IPI Bens de Consumo} & \cellcolor{gray!6}{} & \cellcolor{gray!6}{-} & \cellcolor{gray!6}{}\\
\textbf{IPI Bens i} &  & + & \\
\cellcolor{gray!6}{IPI bsd} & \cellcolor{gray!6}{} & \cellcolor{gray!6}{+} & \cellcolor{gray!6}{}\\
\textbf{IPI Geral} &  & - & \\
\cellcolor{gray!6}{IPIad} & \cellcolor{gray!6}{} & \cellcolor{gray!6}{+} & \cellcolor{gray!6}{}\\
\addlinespace
\textbf{PIB} &  & + & \\
\cellcolor{gray!6}{Saldo Carteira Crédito RL} & \cellcolor{gray!6}{} & \cellcolor{gray!6}{-} & \cellcolor{gray!6}{}\\
\textbf{Volatilidade da Selic} & + &  & \\
\bottomrule
\end{tabular}
\endgroup{}
\end{table}
\label{tab:exanteb}
\fonte{Desenvolvido a partir das fontes citadas}
\end{table}

Ainda analisando a \autoref{tab:exantea} e a \autoref{tab:exanteb}, dois
estudos chegaram a resultados diferentes para os efeitos da volatilidade da
Selic no \emph{spread ex-ante}. Os efeitos do IPCA foram testados em três estudos,
os dois mais recentes encontraram uma relação indireta com a variável
dependente. Em três estudos que examinaram o IGP, dois encontram relação
direta, sendo que um deles foi repetido em período anterior e encontrou relação
indireta.

\begin{table}
\caption{Resumo de estudos sobre o \emph{spread ex-post} no Brasil}
\begin{table}[H]
\centering\begingroup\fontsize{10}{12}\selectfont

\begin{tabular}[t]{>{\raggedright\arraybackslash}p{4cm}>{\raggedright\arraybackslash}p{3cm}>{\raggedright\arraybackslash}p{3cm}>{\raggedright\arraybackslash}p{3cm}}
\toprule
Variável & GUIMARÃES (2002) & DANTAS (2012) & ALMEIDA (2013)\\
\midrule
\cellcolor{gray!6}{Custos Administrativos} & \cellcolor{gray!6}{} & \cellcolor{gray!6}{} & \cellcolor{gray!6}{+}\\
\textbf{Impostos Indiretos} &  &  & Não significativo\\
\cellcolor{gray!6}{Requerimento de Reserva} & \cellcolor{gray!6}{} & \cellcolor{gray!6}{} & \cellcolor{gray!6}{+}\\
\textbf{Atividade Econômica} &  & + & \\
\cellcolor{gray!6}{Bancos Estrangeiros} & \cellcolor{gray!6}{+} & \cellcolor{gray!6}{} & \cellcolor{gray!6}{}\\
\addlinespace
\textbf{Caixa.Depósitos} & + &  & \\
\cellcolor{gray!6}{Grau Concentração} & \cellcolor{gray!6}{} & \cellcolor{gray!6}{+} & \cellcolor{gray!6}{+}\\
\textbf{Liquidez} &  &  & Não significativo\\
\cellcolor{gray!6}{Market Share} & \cellcolor{gray!6}{} & \cellcolor{gray!6}{-} & \cellcolor{gray!6}{+}\\
\textbf{PIB} &  &  & +\\
\addlinespace
\cellcolor{gray!6}{Receita Serviços} & \cellcolor{gray!6}{} & \cellcolor{gray!6}{} & \cellcolor{gray!6}{-}\\
\textbf{Risco Crédito} &  & + & Não significativo\\
\bottomrule
\end{tabular}
\endgroup{}
\end{table}
\label{tab:expost}
\fonte{Desenvolvido a partir das fontes citadas}
\end{table}

A \autoref{tab:expost} traz o resumo dos estudos empíricos dos determinantes do
\emph{spread ex-post} no Brasil, por meio de modelos econométricos utilizando
regressão. Destaca-se que, entre os estudos, dois encontraram significância de
influência direta com o grau de concentração e o \emph{spread} ex-post. E dois dos
estudos chegaram a resultados opostos para os de posição de market share e a
variável dependente.

Este capítulo verificou os principais conceitos, características e estudos
acerca do \emph{spread} bancário no Brasil, identificando as óticas de análise por
evolução, composição e determinantes através da abrangência da amostra,
conteúdo e origem da informação. E que as maiores limitações estão na
dificuldade de desagregação de informações para uma análise mais aprofundada.

No próximo capítulo, será descrita a metodologia de trabalho com a formulação
das hipóteses baseado nas informações e levantamentos dos capítulos anteriores,
nos estudos pesquisados e na teoria econômica, através da coleta, tratamento e
análise de dados.

\chapter{Metodologia}

\chapter{Aplicação}

\chapter{Resultados}

\phantompart

\chapter*[Conclusão]{CONSIDERAÇÕES FINAIS}
\addcontentsline{toc}{chapter}{CONSIDERAÇÕES FINAIS}

\postextual

\addtocontents{toc}{\vspace{-2pt}}

\postextual

\addtocontents{toc}{\vspace{-2pt}}

\ifthenelse{\equal{\terApendice}{Sim}}
{\begin{apendicesenv}

\renewcommand{\thechapter}{\arabic{chapter}}

\chapter{Replicação de Estudos}
\section{}
 
\end{apendicesenv}
}{}

\ifthenelse{\equal{\terAnexo}{Sim}}{
\begin{anexosenv}

\renewcommand{\thechapter}{\arabic{chapter}}
        
\chapter{Cálculo Resultados}

\lipsum[31] 

\lipsum[32] 

\end{anexosenv}
}{}

\ifthenelse{\equal{\terIndiceR}{Sim}}{
\phantompart
\printindex
}{}

\printbibliography

\end{document}
