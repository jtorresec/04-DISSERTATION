% Options for packages loaded elsewhere
\PassOptionsToPackage{unicode}{hyperref}
\PassOptionsToPackage{hyphens}{url}
%
\documentclass[
  ignorenonframetext,
  aspectratio=169,
  ignorenonframetext]{beamer}
\usepackage{pgfpages}
\setbeamertemplate{caption}[numbered]
\setbeamertemplate{caption label separator}{: }
\setbeamercolor{caption name}{fg=normal text.fg}
\beamertemplatenavigationsymbolsempty
% Prevent slide breaks in the middle of a paragraph
\widowpenalties 1 10000
\raggedbottom
\setbeamertemplate{part page}{
  \centering
  \begin{beamercolorbox}[sep=16pt,center]{part title}
    \usebeamerfont{part title}\insertpart\par
  \end{beamercolorbox}
}
\setbeamertemplate{section page}{
  \centering
  \begin{beamercolorbox}[sep=12pt,center]{part title}
    \usebeamerfont{section title}\insertsection\par
  \end{beamercolorbox}
}
\setbeamertemplate{subsection page}{
  \centering
  \begin{beamercolorbox}[sep=8pt,center]{part title}
    \usebeamerfont{subsection title}\insertsubsection\par
  \end{beamercolorbox}
}
\AtBeginPart{
  \frame{\partpage}
}
\AtBeginSection{
  \ifbibliography
  \else
    \frame{\sectionpage}
  \fi
}
\AtBeginSubsection{
  \frame{\subsectionpage}
}
\usepackage{amsmath,amssymb}
\usepackage{lmodern}
\usepackage{iftex}
\ifPDFTeX
  \usepackage[T1]{fontenc}
  \usepackage[utf8]{inputenc}
  \usepackage{textcomp} % provide euro and other symbols
\else % if luatex or xetex
  \usepackage{unicode-math}
  \defaultfontfeatures{Scale=MatchLowercase}
  \defaultfontfeatures[\rmfamily]{Ligatures=TeX,Scale=1}
\fi
% Use upquote if available, for straight quotes in verbatim environments
\IfFileExists{upquote.sty}{\usepackage{upquote}}{}
\IfFileExists{microtype.sty}{% use microtype if available
  \usepackage[]{microtype}
  \UseMicrotypeSet[protrusion]{basicmath} % disable protrusion for tt fonts
}{}
\makeatletter
\@ifundefined{KOMAClassName}{% if non-KOMA class
  \IfFileExists{parskip.sty}{%
    \usepackage{parskip}
  }{% else
    \setlength{\parindent}{0pt}
    \setlength{\parskip}{6pt plus 2pt minus 1pt}}
}{% if KOMA class
  \KOMAoptions{parskip=half}}
\makeatother
\usepackage{xcolor}
\IfFileExists{xurl.sty}{\usepackage{xurl}}{} % add URL line breaks if available
\IfFileExists{bookmark.sty}{\usepackage{bookmark}}{\usepackage{hyperref}}
\hypersetup{
  pdftitle={OS EFEITOS DOS COMPONENTES DO SPREAD EX-POST NA RENTABILIDADE DAS INSTITUIÇÕES BANCÁRIAS},
  pdfauthor={JACKSON DA SILVA TORRES},
  hidelinks,
  pdfcreator={LaTeX via pandoc}}
\urlstyle{same} % disable monospaced font for URLs
\newif\ifbibliography
\setlength{\emergencystretch}{3em} % prevent overfull lines
\providecommand{\tightlist}{%
  \setlength{\itemsep}{0pt}\setlength{\parskip}{0pt}}
\setcounter{secnumdepth}{-\maxdimen} % remove section numbering
\usepackage[utf8]{inputenc} % codificacao de caracteres
\usepackage[T1]{fontenc}    % codificacao de fontes
\usepackage[brazil]{babel}  % idioma
\usepackage{graphicx}       %fundo
\usetheme{default}          % tema
\usecolortheme{orchid}     % cores
\usefonttheme[onlymath]{serif} % fonte modo matematico
\usepackage{wallpaper}

\usepackage{tikz}

\usepackage[
style=abnt,
backref=true,
backend=biber,
maxcitenames=2,
%citecounter=true,
backrefstyle=three,
%nohashothers=true
]{biblatex}

%\setbeameroption{show notes}
%\setbeameroption{show notes on second screen=right}

%\usepackage[pdftex]{graphicx}

\usepackage[final]{pdfpages}

\beamertemplatenavigationsymbolsempty
\hypersetup{pdfpagemode=FullScreen}

%Título
\makeatletter
\newcommand\titlegraphicii[1]{\def\inserttitlegraphicii{#1}}
\titlegraphicii{}
\setbeamertemplate{title page}
{
  \vbox{}
   {\usebeamercolor[fg]{titlegraphic}\inserttitlegraphic\hfill\inserttitlegraphicii\par}
  \begin{centering}
    
    \begin{tikzpicture}[remember picture,overlay]
  \centering
    	\node[anchor=south west, yshift= -1.5mm, xshift=-1.5mm] at 
		(current page.south west) 
		{\includegraphics[width=\paperwidth]{CapaUFPR.jpg}};
    \end{tikzpicture}

    \begin{beamercolorbox}[sep=2pt,center]{institute}
      \usebeamerfont{institute}\insertinstitute
    \end{beamercolorbox}
    \vskip0.1em\par
    
    \begin{beamercolorbox}[sep=6.5pt,center]{author}
      \usebeamerfont{author}\insertauthor
    \end{beamercolorbox}
    
    \begin{beamercolorbox}[sep=6.5pt,center]{title}
      \usebeamerfont{title}\inserttitle\par%
      \ifx\insertsubtitle\@empty%
      \else%
        \vskip0.9em%
        {\usebeamerfont{subtitle}\usebeamercolor[fg]{subtitle}\insertsubtitle\par}%
      \fi%     
    \end{beamercolorbox}%
    %
    \begin{beamercolorbox}[sep=4pt,center]{date}
      \usebeamerfont{date}\insertdate
    \end{beamercolorbox} \vskip3cm
    \vskip3cm
  \end{centering}
  %\vfill
}
\makeatother

\title{}
\subtitle{}
\author{\textbf{}}
\institute{UNIVERSIDADE FEDERAL DO PARANÁ \\ 
DEPARTAMENTO DE ECONOMIA \\
PROGRAMA PROFISSIONAL DE PÓS-GRADUAÇÃO EM ECONOMIA \\
MESTRADO PROFISSIONAL EM ECONOMIA} % opcional
\date{}

\AtBeginSubsection[]
{
  \begin{frame}<beamer>{Outline}
    \tableofcontents[currentsection,currentsubsection]
  \end{frame}
}

\usebackgroundtemplate{
  \centering
  \includegraphics[height=\paperheight]{FundoUFPR2.png}
 }
 
\usepackage{helvet}
\renewcommand{\familydefault}{\sfdefault}
\makeindex
\DeclareUnicodeCharacter{0301}{******}
\DeclareUnicodeCharacter{0303}{******}
\DeclareUnicodeCharacter{0327}{******}
\DeclareUnicodeCharacter{0302}{******}
\ifLuaTeX
  \usepackage{selnolig}  % disable illegal ligatures
\fi

\title{OS EFEITOS DOS COMPONENTES DO \emph{SPREAD} \emph{EX-POST} NA
RENTABILIDADE DAS INSTITUIÇÕES BANCÁRIAS}
\author{JACKSON DA SILVA TORRES}
\date{2021}

\begin{document}
\frame{\titlepage}

\begin{frame}{\textbf{REFERÊNCIAS}}
\protect\hypertarget{referuxeancias}{}
ALMEIDA, F. D. Determinantes do spread bancário ex-post no Brasil: uma
análise de fatores micro e macroeconômicos. Brasíli: Universidade
Católica de Brasíl, 2013.

ALMONACID, Ruben D.; PASTORE, Affonso Celso. Uma nota sobre o
multiplicador da oferta monetária. Pesq. Plan. Econômico, IPEA, Rio de
Janeiro, 1976. Disponível em: \textless http:
//repositorio.ipea.gov.br/bitstream/11058/6813/1/PPE\_v6\_n2\_Uma\%20nota.pdf\textgreater.

ANDREWS, Donald W.K.; LU, Biao. Consistent model and moment selection
procedures for GMM estimation with application to dynamic panel data
models. Journal of Econometrics, MI, USA, 12 jul. 2000.
\end{frame}

\begin{frame}{\textbf{REFERÊNCIAS}}
\protect\hypertarget{referuxeancias-1}{}
ARELLANO, Manuel; BOND, Stephen. Some Tests of Specification for Panel
Data: Monte Carlo Evidence and an Application to Employment Equations.
Review of Economic Studies, v. 58, p.~277--297, 1991. Disponível em:
\textless{}\url{http://people.stern.nyu.edu/wgreene/}
Econometrics/Arellano-Bond.pdf\textgreater.

ARONOVICH, Selmo. Uma nota sobre os efeitos da inflação e do nível de
atividade sobre o spread bancário. Revista Brasileira de Economia, v.
48, n.~1, p.~125--40, 1994.

ASSAF NETO, Alexandre. Finanças corporativas e valor. São Paulo: Atlas,
2020. BABA, Y.; HENDRY, D. F.; STARR, R. M. The Demand for Ml in the USA
1960-1988.
\end{frame}

\begin{frame}{\textbf{REFERÊNCIAS}}
\protect\hypertarget{referuxeancias-2}{}
\_\_\_. Juros e Spread Bancário no Brasil. Brasília, 1999. Disponível
em: \url{https://www.bcb.gov.br/ftp/juros-spread1.pdf}. {[}S.l.{]},
1999. Brasília, 2000. Disponível em:
\textless{}\url{https://www.bcb.gov.br/ftp/}
jurospread112000.pdf\textgreater.

\_\_\_\_. Programação Monerária 2019. {[}S.l.{]}, 2019. Disponível em:
\url{https://www.bcb.gov.br/content/publicacoes/programacaomonetaria/pm-022019p.pdf}.

BACEN, SGS - Sistema Gerenciador de Séries Temporais. Agregados
Monetários. {[}S.l.: s.n.{]}. Disponível em:
\url{https://www3.bcb.gov.br/sgspub/consultarmetadados/consultarMetadadosSeries.do}?
method=consultarMetadadosSeriesInternet\&hdOidSerieSelecionada=27789.
Acesso em: 30/11/2020.
\end{frame}

\begin{frame}{\textbf{REFERÊNCIAS}}
\protect\hypertarget{referuxeancias-3}{}
BACEN, Sistema Gerenciador de Séries Temporais. Base Monetária. {[}S.l.:
s.n.{]}. Disponível em
\url{https://www3.bcb.gov.br/sgspub/consultarmetadados/}
consultarMetadadosSeries.do?method=consultarMetadadosSeriesInternet.
Acesso em: /12/2020.

\_\_\_\_\_. Meios de Pagamentos Ampliados. {[}S.l.: s.n.{]}. Disponível
em:
\url{https://www3.bcb.gov.br/sgspub/consultarmetadados/consultarMetadadosSeries.do}?
method=consultarMetadadosSeriesInternet\&hdOidSerieSelecionada=27810.
Acesso em 12/05/2020.

BANK, WORLD; IMF. Financial sector Assessment: a handbook. Washington
DCo: The World Bank, 2005. Disponível em:
\textless{}\url{http://documents1.worldbank.org/curated/en/}
306701468337879923/pdf/337970rev0Fina10Assessment01PUBLIC1.pdf\textgreater.
\end{frame}

\begin{frame}{\textbf{REFERÊNCIAS}}
\protect\hypertarget{referuxeancias-4}{}
BINDER, Michael; HSIAO, Cheng; PESARAN, M. Hashem. Estimation and
Inference in Short Panel Vector Autoregressions with Unit Roots and
Cointegration. Econometric Theory, Cambridge University Press, v. 21,
n.~4, p.~795--837, 2005. ISSN 02664666, 14694360. Disponível em:
\url{http://www.jstor.org/stable/3533397}.

BLUNDELL, Richard; BOND, Stephen. Initial conditions and moment
restrictions in dynamic panel data models. Journal of Econometrics, v.
87, n.~1, p.~115--143, 1998. Disponível em:
\url{https://www.ucl.ac.uk/~uctp39a/Blundell-Bond-1998.pdf}.

BONTEMPI, Maria Elena; MAMMI, Irene. Implementing a strategy to reduce
the instrument count in panel GMM. The Stata Journal, v. 15, n.~4,
p.~1075--1097, 12 jul. 2000.
\end{frame}

\begin{frame}{\textbf{REFERÊNCIAS}}
\protect\hypertarget{referuxeancias-5}{}
BORDO, M. D; L., JONUNG; L., SIKLOS P. Institutional and the Velocity of
Money: a century of evidence. Economic Inquiry, p.~710--724, 1997.

BRASIL. CONSTITUIÇÃO DA REPÚBLICA FEDERATIVA DO BRASIL DE 1988. Diário
Oficial da República Federativa do Brasil, Brasília, DF, 5 out. 1988.
Disponível em:
\url{http://www.planalto.gov.br/ccivil_03/constituicao/constituicao.htm}.
Acesso em: 7 set. 2020.

\_\_\_\_. DECRETO No 1.455. Diário Oficial da República Federativa do
Brasil, Brasília, DF, 30 dez. 1905. Disponível em:
\textless{}\url{https://www2.camara.leg.br/legin/fed/decret/1900-1909/decreto-1455-30-dezembro-}
1905-582773-publicacaooriginal-105568-pl.html\textgreater. Acesso em: 5
set. 2020.
\end{frame}

\begin{frame}{\textbf{REFERÊNCIAS}}
\protect\hypertarget{referuxeancias-6}{}
\_\_\_\_\_. DECRETO No 14.728. Diário Oficial da República Federativa do
Brasil, Brasília, DF, 16 mar. 1921. Disponível em:
\textless{}\url{https://www2.camara.leg.br/legin/fed/decret/1920-1929/decreto-14728-16-marco-}
1921-504798-publicacaooriginal-1-pe.html\textgreater. Acesso em: 15 mar.
2021. \_\_\_\_\_. DECRETO-LEI No 759, DE 12 DE AGOSTO DE 1969. Diário
Oficial da República Federativa do Brasil, Brasília, DF, 12 ago. 1969.
Disponível em:
\url{http://www.planalto.gov.br/ccivil_03/decreto-lei/del0759.htm}.
Acesso em: 4 set. 2020.
\end{frame}

\begin{frame}{\textbf{REFERÊNCIAS}}
\protect\hypertarget{referuxeancias-7}{}
\_\_\_\_\_. Lei no 4.595, de 31 de dezembro de 1964. Diário Oficial da
República Federativa do Brasil, Brasília, DF, 31 dez. 1964. Disponível
em: \url{http://www.planalto.gov.br/ccivil_03/leis/L4595.htm}. Acesso
em: 4 set. 2020.

BROCK, Philip L.; ROJAS SUAREZ, Liliana. Understanding the behavior of
bank spreads in Latin America. Journal of Development Economics, v. 63,
n.~1,p.~113--134, 2000. Disponível em:
\url{https://EconPapers.repec.org/RePEc:eee:deveco:v:63:y:2000:i:1:p:113-134}.

CAMARGO, Patrícia Olga. A evolução recente do setor bancário no Brasik.
São Paulo: Cultura Acadêmica, 2009.
\end{frame}

\begin{frame}{\textbf{REFERÊNCIAS}}
\protect\hypertarget{referuxeancias-8}{}
CAMPELLO, Mauro Luiz Costa; BRUNSTEIN, Israel. UMA ANÁLISE DA
COMPETITIVIDADE DOS BANCOS DE VAREJO NO BRASIL. REVISTA GESTÃO DA
PRODUÇÃO OPERAÇÕES E SISTEMAS, Unesp, v. 1, n.~1, p.~83--99, 2005.

CARDOSO, Renato Fragelli; KOYAM, Sérgio Mikioa. A CUNHA FISCAL SOBRE A
INTERMEDIAÇÃO FINANCEIRA. In: {[}s.l.{]}: Banco Central do Brasil, 1999.
P. 129--158.

CMN. Resolução CMN 2.624, de 1999. Diário Oficial da República
Federativa do Brasil, Brasília, DF, 29 jul. 1999. Disponível em:
\textless{}\url{https://www.bcb.gov.br/pre/normativos/busca/downloadNormativo.asp?arquivo=}
/Lists/Normativos/Attachments/45083/Res\_2624\_v1\_O.pdf\textgreater.
Acesso em: 4 set. 2020.
\end{frame}

\begin{frame}{\textbf{REFERÊNCIAS}}
\protect\hypertarget{referuxeancias-9}{}
\_\_\_\_\_. Resolução CMN 3.426, de 2006. Diário Oficial da República
Federativa do Brasil, Brasília, DF, 26 dez. 2006. Disponível em:
\url{https://www.bcb.gov.br/pre/normativos/res/1976/pdf/res_0394_v13_P.pdf}.
Acesso em: 4 set. 2020.

\_\_\_. Resolução CMN 394, de 1976. Diário Oficial da República
Federativa do Brasil, Brasília, DF, 20 out. 1976. Disponível em:
\textless{}\url{https://www.bcb.gov.br/pre/normativos/res/1976/pdf/res_0394_v13_P}
pdf\textgreater. Acesso em: 4 set. 2020.

\_\_\_\_. Resolução No 1.524 de 1988. Diário Oficial da República
Federativa do Brasil, Brasília, DF, 24 set. 1988. Disponível em:
\url{https://www.bcb.gov.br/pre/normativos/res/1988/pdf/res_1524_v8_P.pdf}.
Acesso em: 24 fev. 2017.
\end{frame}

\begin{frame}{\textbf{REFERÊNCIAS}}
\protect\hypertarget{referuxeancias-10}{}
\_\_\_\_\_. Resolução no 2.099. Diário Oficial da República Federativa
do Brasil, Brasília DF, 17 ago. 1994. Disponível em:
\url{https://www.bcb.gov.br/pre/normativos/res/1994/pdf/res_2099_v1_O.pdf}.
Acesso em: 4 set. 2020.

CÓRDOBA, Miguel. Análisis Financiero de los Mercados Monetarios y de
Valores. Madrid: Saraiva, 1996.

COSTA, Ana Carla Abrão; NAKANE, Márcio I. Spread bancário no Brasil:
dois esclarecimentos e duas constatações. Tecnologia de Crédito, 2004.
\end{frame}

\begin{frame}{\textbf{REFERÊNCIAS}}
\protect\hypertarget{referuxeancias-11}{}
COSTA NETO, Yttrio Corrêa da. Bancos oficiais no Brasil: origem e
aspectos de seu desenvolvimento. Brasília: Banco Central do Brasil,
2004. Disponível em:
\url{https://www.bcb.gov.br/htms/public/BancosEstaduais/livros_bancos_oficiais.pdf}.

COUTO, Rodrigo Luís Rosa. Metodologia de avaliação da capacidade de
geração de resultados de instituições financeiras. {[}S.l.{]}, 2002.
(Notas Técnicas do Banco Central do Brasil). Disponível em:
\textless Dispon\%C3\%ADvel\%20em\%20\%5Curl\%7Bhttps://www.bcb.gov.br/pec/notastecnicas/port/2002nt26avalgeracaoresultp.pdf\%7D\%
20Acesso\%20em:\%2020/11/2020\textgreater.

DANTAS, José A. Determinantes do spread bancário ex post no mercado
brasileiro. REV. ADM. MACKENZIE, UNIVERSIDADE PRESBITERIANA MACKENZIE,
v. 13, n.~4, p.~48--74, 2012.
\end{frame}

\begin{frame}{\textbf{REFERÊNCIAS}}
\protect\hypertarget{referuxeancias-12}{}
DEMIRGÜÇ-KUNT, Ash; HUIZINGAGA, Harry. Determinants of commercial bank
interest margins and profitability: some international evidence. The
World Bank Economic Review, v. 13, p.~379--408, 1 mai. 1999. Disponível
em: \textless https:
//citeseerx.ist.psu.edu/viewdoc/download?doi=10.1.1.194.3108\&rep=rep1\&type=pdf\textgreater.
Acesso em: 8 set. 2020.

DICK, Astrid. Banking Spreads in Central America: Evolution, Structure,
and Behavior. HIID Development Discussion Papers, Harvard Institute for
International Development, Cambridge, 1999.

DURIGAN, Junior et al.~Fatores macroeconômicos, indicadores industriais
e o spread bancário no Brasil. Revista de Ciências da Administração -
RCA, 2018. DOI: 10.5007/2175-8077.2018v20n51p26.
\end{frame}

\begin{frame}{\textbf{REFERÊNCIAS}}
\protect\hypertarget{referuxeancias-13}{}
FARIA, Miguel Figueira de Faria; MENDES, José Amado. Dicionário de
História Empresarial Portuguesa, Séculos XIX e XX. Porto: INCM, 2014. v.
I.

FIPECAFI. ESTUDO SOBRE A APURAÇÃO DO SPREAD DA INDÚSTRIA BANCÁRIA.
{[}S.l.{]}, 2005. Disponível em:
\url{https://www.bcb.gov.br/ftp/jurospread112000.pdf}.

FRIEDMAN, M.; SCHWARTZ, A. J. A monetary history of the United States,
1867 -- 1960. Princeton: Princeton University Press, 1963.
\end{frame}

\begin{frame}{\textbf{REFERÊNCIAS}}
\protect\hypertarget{referuxeancias-14}{}
\_\_\_\_\_. Monetary Trends in the United States and the United Kingdom
Their Relation to Income, Prices, and Interest Rates, 1867-1975.
Chicago: University of Chicago Press for NBER, 1982.

GRAHAM, Benjamin; MEREDITH, Spencer B. A interpretação das demonstrações
financeiras. 3. ed.~São Paulo: Saraiva, 2012.

GUIMARÃES, Carlos Gabriel. O Estado Imperial brasileiro e os bancos
estrangeiros: o caso do London and Brazilian Bank (1862-1871). Anais do
XXVI Simpósio Nacional de História -- AUHNP São Paulo, julho 2011, 2011.
Disponível em: \textless{}\url{http://www.snh2011.anpuh.org/resources}
/anais/14/1298818435\_ARQUIVO\_ TextoLBBnovo.pdf\textgreater.
\end{frame}

\begin{frame}{\textbf{REFERÊNCIAS}}
\protect\hypertarget{referuxeancias-15}{}
HAFER, R. W.; JANSEN, D. W. The Demand for Money in the United States:
Evidence from Cointegration Test. Journal of Money, Credit, and Banking,
p.~155--68, 1991.

HENDRY, David F.; MIZON, Grayham E. Serial Correlation as a Convenient
Simplification, Not a Nuisance: A Comment on a Study of the Demand for
Money by the Bank of England. The Economic Journal, {[}Royal Economic
Society, Wiley{]}, v. 88,n.~351, p.~549--563, 1978. ISSN 00130133,
14680297. Disponível em: \url{http://www.jstor.org/stable/2232053}.HILL,
R. Carter. Economertia. 3. ed.~São Paulo: Saraiva, 2010.

HO, Thomas S. Y.; SAUNDERS, Anthony. The Determinants of Bank Interest
Margins: Theory and Empirical Evidence. Journal of Financial and
Quantitative Analysis,v. 16, n.~4, p.~581--600, 1981. Disponível
em:\url{https://EconPapers.repec.org/RePEc:cup:jfinqa:v:16:y:1981:i:04:p:581-600_00}.
\end{frame}

\begin{frame}{\textbf{REFERÊNCIAS}}
\protect\hypertarget{referuxeancias-16}{}
HOLTZ-EAKIN, D.; NEWEY, W.; ROSEN, H.S. Estimating vector
autoregressions with panel data. Econometrica, v. 56, n.~1371-1395,
1988. Disponível em: \textless6\textgreater.

JAMES, Gareth et al.~An Introduction to Statistical Learning. 8. ed.~New
York: Springer, 2017.

KAPETANIOS, G. A bootstrap procedure for panel data sets with many
cross-sectional units. The Econometrics Journal, {[}Royal Economic
Society, Wiley{]}, v. 11, n.~2,p.~377--395, 2008. ISSN 13684221,
1368423X. Disponível em: \url{http://www.jstor.org/stable/23116081}.
\end{frame}

\begin{frame}{\textbf{REFERÊNCIAS}}
\protect\hypertarget{referuxeancias-17}{}
KLEIN, Michael A. A Theory of the Banking Firm. Journal of Money, Credit
and Banking, Ohio State University Press, v. 3, n.~2, p.~205--218, mai.
1971. Disponível em: \url{http://www.jstor.org/stable/1991279}.

LEITE, J. C. Tecnologia e organizações: um estudo sobre os efeitos da
introdução de novas tecnologias no setor bancário brasileiro. 1996. Tese
(Doutorado) -- São Paulo.

LEVINE, Ross. Financial Development and Economic Growth: Views and
Agenda. Journal of Economic Literature, American Economic Association,
v. 35, n.~2,(p.~688--726, 1997. ISSN 00220515. Disponível em:
\url{http://www.jstor.org/stable/2729790}.
\end{frame}

\begin{frame}{\textbf{REFERÊNCIAS}}
\protect\hypertarget{referuxeancias-18}{}
MAFFILI, Dener William; BRESSAN, Aureliano Angel; SOUZA, Antônio Artur
da. Estudo da Rentabilidade dos Bancos Brasileiros de Varejo no Período
de 1999 a 2005. Contabilidade Vista e Revista, SI, v. 238, n.~2,
p.~117--138, 12 mai. 2009. Disponível em:
\textless Dispon\%C3\%ADvel\%20em\%20\%5Curl\%7Bhttps:
//www.imf.org/external/pubs/ft/op/238/index.htm\%7D\%20Acesso\%20em:
\%2017\%20fev.\%202021\textgreater.

MAGALHÃES-TIMOTIO, João G. RELAÇÃO ENTRE INDICADORES CONTÁBEIS E O
SPREAD EX-POST DOS BANCOS BRASILEIROS. RACEF -- Revista de
Administração, Contabilidade e Economia da Fundace, v. 9, n.~2,
p.~31--44, 2018.

MATOS, Orlando Carneiro de. Inter-relações entre Desenvolvimento
Financeiro, Exportações e Crescimento Econômico: Análise da Experiência
Brasileira. In: NOTAS Técnicas do Banco Central do Brasil. Brasília:
BCB, 2003. Disponível em:
\textless{}\url{https://www.bcb.gov.br/content/publicacoes/notastecnicas/2003nt40Inter-}
relentreDesenvFinanp.pdf\textgreater.
\end{frame}

\begin{frame}{\textbf{REFERÊNCIAS}}
\protect\hypertarget{referuxeancias-19}{}
MILLER, Stephen. Monetary Dynamics: An Application of Cointegration and
Error-Correction Modeling. Journal of Money, Credit and Banking, v. 23,
n.~2, p.~139--54, 1991. Disponível em:
\url{https://EconPapers.repec.org/RePEc:mcb:jmoncb:v:23:y:1991:i:2:p:139-54}.

NEVES JÚNIOR, Idalberto José das; SOARES RIBEIRO, Francilanes;(MENDES,
Frederico. EFICIÊNCIA OPERACIONAL: UMA ANÁLISE EXPLORATÓRIA DOS 50
MAIORES BANCOS BRASILEIROS PELO RANKING BACEN. 4o Congresso USP de
Iniciação Científica em Contabilidade, USP, 2007. Disponível em:
\url{https://intercostos.org/documentos/apellidos/Das\%20Neves\%201.pdf}.
OREIRO, José Luís da Costa; PAULO, Luiz Fernando de. Determinantes
macroeconômicos do spread bancário no Brasil: teoria e evidência
recente. Economia Aplicada, v. 10, n.~4, p.~609--634, 2006.
\end{frame}

\begin{frame}{\textbf{REFERÊNCIAS}}
\protect\hypertarget{referuxeancias-20}{}
REY, Letícia Dias. Spread Bancário Brasileiro: um indicador de excessos?
Insper, 2017. Disponível em:
\textless{}\url{http://dspace.insper.edu.br/xmlui/bitstream/handle/11224/1799/Let\%}
C3\%ADcia\%20Dias\%20Rey\_Trabalho.pdf?sequence=1\textgreater.

ROVER, Suliani; TOMAZZIA, Eduardo Cardeal; FÁVER, Luiz Paulo.
Determinantes Econômico-Financeiros e Macroeconômicos da Rentabilidade:
Evidências Empíricas do Setor Bancário Brasileo. Revista Brasileira de
Economia, XXXV Encontro da ANPAD, 2011.

SARGAN, Denis J. Wages and Prices in the U.K.: A Study in Econometric
Methodology. in Econometric Analysis for National Economic Planning,
Paul Hart, Gary Mills and John K. Whitaker (eds) Colston Papers1,
Butterworths, London, v. 16, p.~25--54, 1964.
\end{frame}

\begin{frame}{\textbf{REFERÊNCIAS}}
\protect\hypertarget{referuxeancias-21}{}
SIGMUND, Michael; FERSTL, Robert. Panel Vector Autoregression inR with
the Package panelvar, 2008. Disponível em:
\textless{}\url{https://www.researchgate.net/publication/}
312165764\_Panel\_Vector\_Autoregression\_in\_R\_with\_the\_Package\_Panelvar\textgreater.

SINGH, Anoop et al.~Stabilization and Reform in Latin America: A
Macroeconomic Perspective on the Experience Since the Early 1990s.
Occasional Paper, International Monetary Fund, v. 238, n.~2, fev. 2005.
Disponível em:
\url{https://www.imf.org/external/pubs/ft/op/238/index.htm}.

SOUZA, Rodrigo Mendes Leal de. Estrutura e determinantes do spread
bancário no Brasil:uma resenha comparativa da literatura empírica. Rio
de Janeiro: Universidade do Estado do Rio de Janeiro, 2006.
\end{frame}

\begin{frame}{\textbf{REFERÊNCIAS}}
\protect\hypertarget{referuxeancias-22}{}
VASCONCELLOS, Marco Antonio Sandoval de. Macroeconomia. 3. ed.~São
Paulo: Atlas, 2001.

VIEIRA, Heleno Piazentini; PEREIRA, Pedro Luiz Valls. Velocidade da
moeda e ciclos econômicos no Brasil, 1900-2016. Insper Working Paper,
New York, NY, p.~1--38, 2016. Disponível em:
\textless{}\url{https://www.insper.edu.br/wp-content/uploads/2018/10/Velocidade-}
moeda-ciclos-economicos-Brasil.pdf\textgreater.

ZIVOT, Eric; WANG, Jiahui. Vector Autoregressive Models for Multivariate
Time Series. In: Modeling Financial Time Series with S-Plus. Springer,
New York, NY, p.~369--413, 2003. Disponível em:
\url{https://doi.org/10.1007/978-0-387-21763-5_11}.
\end{frame}

\end{document}
